\documentclass[ ]{standalone}
\usepackage[utf8]{inputenc}
\usepackage{amsmath}
%\usepackage[mathscr]{eucal}
\usepackage{amsthm}
\usepackage{amssymb}
\usepackage{bm}
\usepackage{array}
\usepackage[T1]{fontenc}
\usepackage[english]{babel}
\usepackage{float}
\usepackage{enumerate}
\usepackage{graphicx}
\usepackage{color}
\usepackage{tikz}
\usetikzlibrary{tikzmark}

%%%%%%%%%%%%%%%%%%%%%%%%%%%%%%%%%%%%%%%%%%%%%%
\usepackage{epstopdf}
\usepackage{latexsym}
\usepackage{keyval}
\usepackage{ifthen}
\usepackage{moreverb}
%\usepackage{mathastext}
\usepackage[shell,cleanup,subfolder]{gnuplottex}
%%%%%%%%%%%%%%%%%%%%%%%%%%%%%%%%%%%%%%%%%%%%%%

%\definecolor{myBlue}{RGB}{0,84, 159}
%\usepackage[lmargin=2cm,rmargin=0cm,tmargin=2cm,bmargin=2cm]{geometry}
\usepackage[pass,paperwidth=8.5in,paperheight=11in]{geometry}
%\geometry{
%	paperwidth=30cm,
%	paperheight=0cm,
%	margin=2cm
%}

 \renewcommand{\familydefault}{\sfdefault}


\begin{document} 
%\begin{minipage}{800pt}
% \hspace{1.5cm}\resizebox{750pt}{!}{\input{pp/diag}}\hspace*{0.550cm}\\[40pt]
%\end{minipage}


\begin{minipage}{15in}
	\vspace*{0.7cm}
	\hspace*{1.5cm}
\begin{gnuplot}[terminal={epslatex},scale=0.9,terminaloptions={color size 15.75,2.5 rounded}]
        set border lw 2
        unset grid
        unset key
        set loadpath '~/Documents/Data/UPO-tACs/7_results/data-overview/' '~/Documents/Data/UPO-tACs/1_Raw/'
	set multiplot layout 1,2
	set ytics 1e-6

	set ylabel '\large LFP [V]'
	set xlabel '\large $t$ [s]'
	set yrange[-1.5e-6:1.5e-6]
	plot 'mov_pre.dat' u 1:(-2e-6):(0):(4e-6) w vectors nohead lc 'gray' , 'ts_ch100.dat' ev 10 w l lc 1 lw 1

	set xrange[200:204]
	set key nobox
	plot 'ts_ch100.dat' ev 10 w l lc 1 t 'Channel 100'


\end{gnuplot}
\vspace*{0.75cm}
\centering
\begin{gnuplot}[terminal={epslatex},scale=0.9,terminaloptions={color size 5.75,3.5 rounded}]

        set border lw 2
        unset grid
        unset key
        set loadpath '~/Documents/Data/UPO-tACs/7_results/data-overview/'

	set xrange[0:200]
	set yrange[1e-18:1e-14]
	set xlabel '\large Frequency [Hz]'
	set ylabel '\large Power [V$^2$]'

	df1=0.019073486328125
	df2=0.00999996000016

	set logs y

	# For mthm is necessary to output the example in a different file with:
	# awk 'NR==100{print $0}' psd_mthm.dat | tr " " "\n" > example_mthm.dat

	set key top right nobox

	plot 'psd_rawhm.dat' u (df1*$0):($100*2*pi) w l lw 0.1 lc 'gray' t 'Plain FFT' ,\
		'example_mthm.dat' u (df2*$0):($1) w l lc 1 t '',\
		'psd_gausshm.dat' u (df1*$0):($100*2*pi) w l lc 2 t 'Gaussian filter',\
		100.0 w l lc 1 t 'Multitaper',\

\end{gnuplot}
\vspace*{0.75cm}
\centering
\begin{gnuplot}[terminal={epslatex},scale=0.9,terminaloptions={color size 15.75,3.5 rounded}]

        set border lw 2
        unset grid
        unset key
        set loadpath '~/Documents/Data/UPO-tACs/7_results/data-overview/'
	set multiplot layout 1,3
		
	set pale @RAINBOW

	set xrange[0:200]
	set yrange[0:384]
	set ylabel '\large Channel'
	set xlabel '\large Frequency [Hz]'

	k=5 ; # skip data to reduce image size
	# ---------------------------------------------------

	df=0.019073486328125
	set logs zcb
	set cbrange[1e-17:1e-14]
	set title '\large Plain FFT'

	plot 'psd_rawhm.dat' matrix ev k u (df*($2-1)):1:($3*2*pi) w ima
	#plot sin(x)

	# ---------------------------------------------------

	set title '\large FFT + Gaussian filter'
	plot 'psd_gausshm.dat' matrix ev k u (df*($2-1)):1:($3*2*pi) w ima
	#plot sin(x)

	# ---------------------------------------------------

	df=0.00999996000016
	set logs zcb
	set title '\large Multitaper'
	plot 'psd_mthm.dat' matrix ev k u (df*$1):2:3 w ima
	#plot sin(x)

	# ---------------------------------------------------

	
\end{gnuplot}
\end{minipage}
\end{document}
